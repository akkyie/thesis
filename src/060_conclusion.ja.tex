\chapter{結論}
\label{chap:conclusion}

\section{本研究のまとめ}
\label{section:conclusion}

本研究では、マイコンにおける実行内容をWebAssemblyで記述することでプログラムの動的な更新が効率的に可能になるという仮説のもと、マイコン向けWebAssembly実行環境を設計した。また、その実現可能性を評価するため、WebAssemblyバイナリのインタプリタをESP32上に実装し、実行速度とメモリフットプリントを計測した。

その結果、定数を返す関数実行で約260倍、加算と関数呼び出しを含む関数実行で約3000倍のオーバーヘッドがあった。
また、メモリフットプリントについて、関数呼び出しの度に208バイトの増加が見られた。

(分析を修正して分かった制約をかく)

\section{今後の課題}

本研究では、静的なWebAssemblyバイナリの実行を検証した。
ネットワーク上から取得し、また実行終了後にネットワーク上からバイナリを更新するためには、通信のための計算負荷・メモリ消費も想定する必要がある。
また、本研究では100バイト未満のバイナリをマイコン上で実行できることを確認したが、より大きなバイナリを実行する際には、ネットワークから流れてくるバイト列を逐一パースし実行するなどの工夫が必要だろう。

また、本研究で実装した実行環境では、ホストプログラムからWebAssemblyプログラムの関数を一方的に呼び出せるのみであったが、入出力を実現するためにはWebAssemblyプログラムからマイコンの機能へのアクセス手段を提供する必要がある。
