\chapter{関連技術}
\label{chap:related_works}

本章では、本研究と同様にマイコン上で実行プログラムをインターネット上から取得する手法について、既存の関連技術を示す。

\section{MicroPython}

MicroPythonは、マイコンを始めとした低性能端末をターゲットとしたPythonコンパイラおよびランタイム実装である\cite{micropython}。
MicroPythonではHTTP通信のライブラリが標準で提供されており、ファイルからのスクリプトの読み込み・実行も行えるため、インターネット上からプログラムを取得し実行すること自体は可能である。

しかし、Pythonで記述されたスクリプトは人間が読み書きするために設計されたものであり、ファイルサイズやパースのためのオーバーヘッドが大きい。
また、Pythonのリファレンス実装であるCPythonは実行高速化のためにバイトコード({\tt pyc}ファイル)へのコンパイルを行うが、このバイトコードの形式はCPythonの内部仕様とされており、形式化された定義は存在しない\cite{python_bytecode}。

\section{ESP-IDFによるOTA}

Espressif SystemsによるESP8266およびESP32向け開発環境であるESP-IDF\cite{esp_idf}は、OTAアップデートを行う手段を提供している\cite{esp_ota}。
起動するプログラムが2つ格納できるようにパーティションを区切り、片方を起動に用いている間にもう片方に格納されているプログラムをアップデートすることで、起動中のOTAアップデートを実現している。

