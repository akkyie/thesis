\begin{jabstract}

本論文では、マイコンにおけるプログラムの動的な更新を効率的に行うために、マイコン向けWebAssembly実行環境を提案する。

WebAssemblyはWebブラウザでの実行を想定して設計された仮想命令セットアーキテクチャである。
Webブラウザにおける標準化された実行形式として利用が広がりつつあり、開発環境による対応も進んでいる。
また、WebAssemblyの命令セットや実行セマンティクスはWebブラウザ以外の環境への応用も想定されて設計されているため、実行形式としてのWebAssemblyの活用はWebブラウザ外にも広がっていくと想定される。
しかしながら、マイコンのような計算資源が限られた環境下におけるWebAssemblyの応用可能性については、いまだ十分に検討されていない。

そこで、本研究では、マイコンにおいてWebAssemblyバイナリをインターネット上からダウンロードし実行するための環境を設計する。
また、その実現可能性を評価するため、ESP32マイコン上にWebAssemblyバイナリのインタプリタを実装した。
その上で、実装した実行環境について、関数の呼び出しにおける実行速度とメモリフットプリントを計測し、ネイティブでの関数実行と比較した。

評価では、ネイティブでの関数実行に比べて定数を返す関数では約280倍、関数呼び出しと加算を含む関数では約3000倍のオーバヘッドがあることが明らかになった。また、関数呼び出しのたびに280バイトのヒープ領域を使用することが分かった。

\end{jabstract}
